\documentclass[10pt,xcolor=dvipsnames]{beamer}
\usetheme[progressbar=frametitle]{metropolis}
\usepackage{booktabs}
\usepackage{amsmath}
\usepackage[scale=2]{ccicons}
\usepackage{colortbl}
\usepackage{array}
\usepackage{ragged2e}
\newcolumntype{P}[1]{>{\RaggedRight\hspace{0pt}}p{#1}}
%\usepackage{graphicx}
\usepackage{graphicx,color,layout}
\usepackage[percent]{overpic}
\usepackage{subfigure}
\usepackage[]{siunitx}
\usepackage{chronology}
\usepackage{xcolor}
\usepackage{array}
\newcolumntype{L}[1]{>{\raggedright\let\newline\\\arraybackslash\hspace{0pt}}m{#1}}
\newcolumntype{C}[1]{>{\centering\let\newline\\\arraybackslash\hspace{0pt}}m{#1}}
\newcolumntype{R}[1]{>{\raggedleft\let\newline\\\arraybackslash\hspace{0pt}}m{#1}}
%\usepackage{subcaption}
%\usepackage{caption}
\usepackage{mwe}
% draw flowchart with TIKZ
\usepackage{tikz}
\usetikzlibrary{shapes,arrows}
%%%%%%%%%%%%%%%%%%%%%%%%%
\usepackage{pgfplots}
\usepgfplotslibrary{dateplot}
\usepackage{natbib}
\usepackage{xspace}
\newcommand{\themename}{\textbf{\textsc{metropolis}}\xspace}

\title{Effect of Homogeneity in Engineering}
\subtitle{Are we engineers or scientists?}
\date{\today}
\author{DR. Eng. Mohamad Reza Salehi Sadaghiani}
\institute{Geotechnical and Hydraulical  Engineering}
\titlegraphic{\hfill\includegraphics[height=0.7cm]{logoOr}}


%%%%%%%%%%%%%%%%%%%%%%%%%%%%%%% navigation symbols
%\setbeamertemplate{navigation symbols}[default]
%\setbeamercolor{navigation symbols}{use=structure,fg=structure.fg!80!bg}
%\setbeamercolor{navigation symbols dimmed}{use=structure,fg=structure.fg!60!bg}


\begin{document}

\maketitle

 \begin{frame}{Table of contents}
   \setbeamertemplate{section in toc}[sections numbered]
   \tableofcontents[hideallsubsections]
 \end{frame}

%%%%%%%%%%%%%%%%%%%%%%%% hypertarget and buttons
% \begin{frame}
% \hyperlink{label}{\beamerbutton{I jump to fourth slide of next frame}}
% \end{frame}

% \begin{frame}
% \begin{itemize}[<+->]
% \item First.
% \item Second.
% \item Third.
% \item Fourth.\hypertarget<4>{label}{\beamerbutton{I'm on the fourth slide}}
% \item Fifth.
% \end{itemize}
% \end{frame}
{
\usenavigationsymbolstemplate{}
\section{A short history of my professional background}
}



{
\begin{frame}{Professional backgraund}
\vspace{-0.60cm}
\begin{itemize}
	\item Education 1 $\Rightarrow$ B.Sc. Civil Eng. Tehran Azad University
	\item Education 2 $\Rightarrow$ M.Sc. Structural Eng. KNT University Tehran
	\item Education 3 $\Rightarrow$ M.Sc. Environmental  Eng. Bauhaus Uni Weimar
	\item Education 3 $\Rightarrow$ Ph.D. Geotechnical hydraulical Eng. Bauhaus Uni Weimar

\end{itemize}
\end{frame]
}



{
\usenavigationsymbolstemplate{}
\section{Relevance of study}
}
{
\usenavigationsymbolstemplate{}
\begin{frame}{Suffusion in granular materials}

\begin{figure}
\includegraphics[trim = 0mm 40mm 0mm 40mm,,clip, width=.95\linewidth]{Dike_with_suffusion.png} 
\vspace{-0.5cm}
       \caption{schematic view of an embankment dam - due to the core overtopping, the support body of the dam comes to suffusion}
        \label{fig:dam_suffusion}
\end{figure}
\vspace{-0.60cm}
\begin{itemize}
	\item Homogeneous arrangement of the soil particles $\Rightarrow$ property improvement \citep{Saucke:1999}
	\item What did we do ? Thicker layers of soil.
	\item \alert{\textbf{There is still report of embankment breakages \citep{bonelli2012erosion}}}.
\end{itemize}
\end{frame}
}
{

\begin{frame}[fragile]{Soil Matrix}

\begin{figure}[H]
   \centering
     \includegraphics[width=.99\linewidth]{soil_matrix_new.png}
        \caption{Different soil matrix classified based on skeletal behavior}
        \label{fig:soil_matrix}
\end{figure}
%\vspace{-0.4cm}
%{Particle arrangement $\;\longrightarrow\;$  behavior of all materials}
\end{frame}

\begin{frame}{PSDs from literature}

\begin{figure}

\includegraphics[trim = 0mm 0mm 0mm 0mm,,clip, height=0.8\textheight]{PSD_stable_DFM.png} 
\vspace{-0.5cm}
       \caption{PSDs of theoretically analyzed soils - stable soils }
        \label{fig:stb}
\end{figure}

\end{frame}
}



\begin{frame}{PSDs from literature}
\begin{figure}
\includegraphics[trim = 0mm 0mm 0mm 0mm,,clip, height=0.8\textheight]{PSD_transitional_TM.png} 
\vspace{-0.5cm}
       \caption{PSDs of theoretically analyzed soils - transitional soils }
        \label{fig:trnas}
\end{figure}

\end{frame}
\begin{frame}{PSDs from literature}
\begin{figure}
\includegraphics[trim = 0mm 0mm 0mm 0mm,,clip, height=0.8\textheight]{PSD_unstable_DCM.png} 
\vspace{-0.5cm}
       \caption{PSDs of theoretically analyzed soils - unstable soils }
        \label{fig:unstab}
\end{figure}
\end{frame}

\begin{frame}{Comparison of different suffusion criteria}

\begin{table}[b]

{
\scriptsize
\begin{center}
\begin{tabular}{lcccccc}
PSD & Sherard & Kenny-Lau & Burenkova & Wan-Fell & Sadaghiani-Witt & Witt\\
& (1979) & (1984, 1985) & (1993) & (2008) & (2012) & (2013)\\\hline
A1  & \textcolor{red}{U} & \textcolor{red}{U}  & \textcolor{Aquamarine}{S} & \textcolor{Aquamarine}{S} & \textcolor{Aquamarine}{S} & \textcolor{Aquamarine}{S}\\
2R  & \textcolor{red}{U} & \textcolor{red}{U}  & \textcolor{red}{U,M}& \textcolor{Aquamarine}{S} & \textcolor{Aquamarine}{S} & \textcolor{Aquamarine}{S} \\
3R  & \textcolor{red}{U} & \textcolor{Aquamarine}{S,M}& \textcolor{Aquamarine}{S}& \textcolor{Aquamarine}{S} & \textcolor{Aquamarine}{S} & \textcolor{Aquamarine}{S} \\
4R  & \textcolor{red}{U} & \textcolor{red}{U}  & \textcolor{Aquamarine}{S} & \textcolor{Aquamarine}{S} & \textcolor{Aquamarine}{S} & \textcolor{Aquamarine}{S} \\
5** & \textcolor{red}{U} & \textcolor{red}{U}  & \textcolor{red}{U,M}& \textcolor{Aquamarine}{S} & \textcolor{Aquamarine}{S} & \textcolor{Aquamarine}{S} \\
6** & \textcolor{red}{U} & \textcolor{red}{U}  & \textcolor{red}{U,M}& \textcolor{Aquamarine}{S} & \textcolor{Aquamarine}{S} & \textcolor{Aquamarine}{S} \\
7** & \textcolor{red}{U} & \textcolor{red}{U}  & \textcolor{red}{U}  & \textcolor{Aquamarine}{S} & \textcolor{Aquamarine}{S} & \textcolor{Aquamarine}{S} \\
RD  & \textcolor{red}{U} & \textcolor{red}{U}  & \textcolor{Aquamarine}{S} & \textcolor{Aquamarine}{S} & \textcolor{Aquamarine}{S} & \textcolor{Aquamarine}{S} \\
10  & \textcolor{red}{U} & \textcolor{Aquamarine}{S}& \textcolor{Aquamarine}{U} & \textcolor{red}{U}   & \textcolor{red}{U,M}  & \textcolor{red}{U}\\
11  & \textcolor{red}{U} & \textcolor{red}{U}   &  \textcolor{Aquamarine}{S}& \textcolor{Aquamarine}{S}&  \textcolor{Aquamarine}{S} & \textcolor{Aquamarine}{S} \\
15**& \textcolor{red}{U}  & \textcolor{red}{U}  & \textcolor{red}{U}    & \textcolor{red}{U}   & \textcolor{red}{U,M} & \textcolor{Aquamarine}{S}\\
9   & \textcolor{red}{U}  & \textcolor{red}{U}  & \textcolor{Aquamarine}{S} & \textcolor{Aquamarine}{S}& \textcolor{red}{U}  & \textcolor{Aquamarine}{S}\\
13**&  \textcolor{red}{U} & \textcolor{red}{U}  & \textcolor{Aquamarine}{S} & \textcolor{Aquamarine}{S}& \textcolor{red}{U} & \textcolor{red}{U} \\
14A**&\textcolor{red}{U} & \textcolor{red}{U}   & \textcolor{red}{U}    & \textcolor{red}{U,M} & \textcolor{red}{U} & \textcolor{red}{U} \\
A2 & \textcolor{red}{U} & \textcolor{red}{U}    & \textcolor{red}{U}    & \textcolor{red}{U}   & \textcolor{red}{U} & \textcolor{red}{U} \\
A3 & \textcolor{red}{U} & \textcolor{red}{U}    & \textcolor{red}{U}    & \textcolor{red}{U}   & \textcolor{red}{U} & \textcolor{red}{U}\\
B1 & \textcolor{red}{U} & \textcolor{red}{U}    & \textcolor{Aquamarine}{S} &\textcolor{red}{U}    & \textcolor{red}{U} & \textcolor{red}{U} \\
B2 & \textcolor{red}{U} & \textcolor{red}{U}    & \textcolor{red}{U}    & \textcolor{red}{U}   & \textcolor{red}{U} & \textcolor{red}{U} \\
C1 & \textcolor{red}{U} & \textcolor{red}{U}    & \textcolor{red}{U}    & \textcolor{red}{U}   & \textcolor{red}{U} & \textcolor{Aquamarine}{S} \\
D1 & \textcolor{red}{U} & \textcolor{red}{U}    & \textcolor{Aquamarine}{S} & \textcolor{red}{U}   & \textcolor{red}{U} & \textcolor{Aquamarine}{S}\\\arrayrulecolor{black}\hline
\end{tabular}
    \label{tab:Internal_assessment_sad2012}  
\end{center}
}%
\end{table}
{\small ** the soil has more than 5\% cohesive fine particles}
\end{frame}

\begin{frame}{Used Particle Size Distribution (PSD)}
\begin{figure}[H]
   \centering
\includegraphics[trim = 0mm 4mm 0mm 17mm, clip, width=.88\linewidth]{PSD_Semar.pdf} F
    \caption{Used soil for suffusion tests - fluviatal soil from river Rhine (mainly consisted of round spherical particles}
    \label{fig:suff_test_psd}
\end{figure}

\end{frame}

\begin{frame}[fragile]{Suffusion test} \label{suffusion}

\begin{columns}
\begin{column}{.46\textwidth}
\begin{figure}
	\centering
		\hyperlink{img1_ST_setup}{   \includegraphics[trim = 0mm 3mm 0mm 0mm, clip, width=.98\linewidth]{Fig4_ST_4L.pdf}   }
	\caption{Setup of suffusion tests}
	\label{fig:Fig4_ST_4L}
\end{figure}

\end{column}%
\begin{column}{.53\textwidth}
\begin{figure}
	\centering
		\hyperlink{img2_Scaterplot}{  \includegraphics[trim = 0mm 3mm 0mm 0mm, clip, width=.98\linewidth]{Scatterplot_ST_EM2.pdf}  }
		\vspace{-0.4cm}
	\caption{scatterplot of results }
	\label{fig:Fig4_ST_4L}
\end{figure}

\end{column}%
\end{columns}
	Requirements for quantification of homogeneity:\\
	\begin{itemize}
	\item \textbf{Requirement 1:} Identification of Soil Matrix
	\item \textbf{Requirement 2:} Determination of Representative Elementary Volume (REV)
	\end{itemize}
\end{frame}

%\begin{frame}[fragile]{Relevance of the study}
%
%\textsc{\textbf{What is necessary ?}}
%\alert{Quantification of homogeneity and including its effect on suffusion!}
%
%Ultimately, an improved homogeneity assessment method $\Rightarrow$ systematic methodologies for soil placement techniques and tool design to get a higher coefficient of homogeneity 
%\end{frame}



%\begin{frame}[fragile]{Research Goals}
%
%\tikzstyle{decision} = [diamond, draw, fill=orange!40, 
%    text width=4em, text badly centered, node distance=3cm, inner sep=0pt]
%\tikzstyle{block} = [rectangle, draw, fill=orange!40, 
% text centered, rounded corners, minimum height=4em, text width=3.5cm]
%\tikzstyle{block1} = [rectangle, draw, fill=olive!40, 
% text centered, rounded corners, minimum height=4em, text width=3.5cm]
%\tikzstyle{block2} = [rectangle, draw, fill=blue!40, 
% text centered, rounded corners, minimum height=4em, text width=2.5cm]
%\tikzstyle{line} = [draw, -latex']
%\tikzstyle{cloud1} = [draw, ellipse,fill=orange!40, node distance=4.3cm,
%    minimum height=2em]
%    \tikzstyle{cloud2} = [draw, ellipse,fill=olive!40, node distance=4.3cm,
%    minimum height=2em]
%{\small                
%\begin{tikzpicture}[node distance = 2cm, auto]
%    % Place nodes
%    \node [block, node distance=2.3cm] (GoalO) {\textbf{Goal 1:}\\ Risk assessment of a widely graded soil against suffusion};
%    \node [block, below of=GoalO, node distance=2cm] (GoalT) {\textbf{Goal 2:}\\ Soil matrix identification};
%    \node [block1, below of=GoalT, node distance=2cm] (GoalThree) {\textbf{Goal 3:}\\ Identification of REV};
%    \node [block1, below of=GoalThree, node distance=2cm] (GoalF) {\textbf{Goal 4:}\\ Homogeneity quantification };
%    \node [block2, right of=GoalThree, node distance=4.2cm] (GoalFi) {\textbf{Final Goal}\\ Suffusion Criterion };
%    
%\node [cloud1, left of=GoalO] (EXP) {\small Experimental};
%\node [cloud2, left of=GoalThree] (NUM) {\small Numerical};
%    % Draw edges
%    \path [line] (GoalO) -- (GoalT);
%    \path [line] (GoalT) -- (GoalThree);
%    \path [line] (GoalThree) -- (GoalF);
%    \path [line] (GoalThree) -- (GoalFi);
%    \path [line] (GoalO) -| (GoalFi);
%    \path [line] (GoalT) -| (GoalFi);
%    \path [line] (GoalF) -| (GoalFi);
%    \path [line,dashed] (EXP) -- (GoalO);
%    \path [line,dashed] (EXP) |- (GoalT);
%    \path [line,dashed] (NUM) -- (GoalThree);
%    \path [line,dashed] (NUM) |- (GoalF);
%\end{tikzpicture}
%}
%\end{frame}

%%%%%%%%%%%%%%%%%%%%%%%%%%%%%
\section{Requirement 1 : Identification of Soil Matrix}
%%%%%%%%%%%%%%%%%%%%%%%%%%%%%

%
% Place your slide code here
%


\begin{frame}[fragile]{Particle arrangement $\;\longrightarrow\;$  behavior of all materials}
\begin{figure}[H]
   \centering
     \includegraphics[trim = 0mm 3mm 0mm 0mm, clip, width=.99\linewidth]{CM_DCM.png}
        \caption{Sequential Fill Test (SFT) - soil matrix DCM}
        \label{fig:EXP-SFT1}
\end{figure}

\end{frame}


%%%%%%%%%%%%%%%%%%%%%%%%%%%%%%%%%%%%%%%%%


%%%%%%%%%%%%%%%%%%%%%%%%%%%%%%%%%%%%%%%%%
{
\usenavigationsymbolstemplate{}

%
% Place your slide code here
%

\begin{frame}[fragile]{The real maximum and minimum density}
%\begin{columns}[T,onlytextwidth]
%\column{0.66\textwidth}
\begin{figure}[H]
\includegraphics[trim = 0mm 78mm 0mm 78mm, clip, width=\textwidth,height=0.82\textheight,keepaspectratio]{Real_Loosest_Densest.pdf} 
\vspace{-0.2cm}      
 \caption{Loosest and densest packings in a poly-disperse material}
        \label{fig:loosedense}
\end{figure}
%\column{0.33\textwidth}
%\end{columns}
\end{frame}

{
\usenavigationsymbolstemplate{}

\begin{frame}%{Coefficient of Homogeneity $C_{H}$ - Experimentally}
	\begin{figure}[H]
\includegraphics[trim = 0mm 2mm 0mm 2mm, clip, width=\textwidth,height=0.99\textheight,keepaspectratio]{SFT-HOMO_SEG.png}      
  %\caption{Experimental measurement of Coefficient of Homogeneity $C_{H}$}
        \label{fig:SFT_Homo_Seg}
\end{figure}
\end{frame}
}

%%%%%%%%%%%%%%%%%%%%%%%%%%%%%%%%%%%%%%%%%
}

{
\usenavigationsymbolstemplate{}
%
% Place your slide code here
%
\section{Requirement 2: Determination of REV}
}
{

\usenavigationsymbolstemplate{}
%
% Place your slide code here
%
\begin{frame}{Establishment of the properties of the porous medium}
\begin{figure}[H]
\centering
\includegraphics[trim = 0mm 68mm 0mm 68mm, clip, width=.82\linewidth]{REV5.pdf} 
\vspace{-0.2cm}
\caption{Schematic graph of how a measured property varies with
sample volume and REV( modified after \cite{bear2012hydraulics})}
\end{figure}
\end{frame}
}
%%%---------------------%%%%%

\begin{frame}{Numerical results for SFT}


\begin{figure}[H]
   \centering
\includegraphics[trim = 25mm 35mm 25mm 30mm, clip,width=.9\linewidth]{Numerical_SFT.pdf} 
        \caption{Comparison of simulated absolute height and SFT results with a mass of $ M $ and cylinder diameter of $ 7 \times D_{max}$ for PSD2}
        \label{fig:Comparison_simulation_Test}
\end{figure}
\end{frame}



{
\usenavigationsymbolstemplate{}
\begin{frame}{Parameter study for finding REV for DCM}

%Proof for Prof. Witt for different effect of homogeneity on particle transport.
{\scriptsize
\begin{table}[b]

\begin{center}
\begin{tabular}{lcccccc}
$D_{cylinder}$             & Experimental Nr.    & Numerical Nr. & Mass\\\hline
$15\times D_{max}$  & -              & 1           &  $M=13.1\; kg$	\\
$12\times D_{max}$  & -              & 1           &  $M=10.8\; kg$	\\
$9\times D_{max}$  & -              & 1           &  $M=8.5\; kg$	\\
$8\times D_{max}$  & -              & 2           &  $M=6.4\; kg$	\\
$7\times D_{max}$  & 3              & 5           &  $M=5.0\; kg$ \\
$7\times D_{max}$  & 3              & 5           &  $M=6.4\; kg$	\\
$7\times D_{max}$  & 3              & 5           &  $M=3.2\; kg$	\\
$7\times  D_{max}$  & 3              & 5           & $M=2.56\; kg$\\
$7\times  D_{max}$  & 3              & 5           &  $M=1.6\; kg$\\
$7\times  D_{max}$  & 3              & 5           &  $M=1.28\; kg$\\
$6\times D_{max}$ & -              & 3            &  $M=3.2\; kg$	\\
$6\times D_{max}$ & -              & 3            &  $M=3.2\; kg$	\\
$5\times D_{max}$ & -              & 3            &  $M=3.2\; kg$	\\
$4\times D_{max}$ & -              & 3            &  $M=3.2\; kg$	\\
$3\times D_{max}$ & -              & 3            &  $M=3.2\; kg$	\\\hline
 $ \sum $ &  15 & 50 & - \\\hline
\end{tabular}
    \label{tab:Values of Volume Flux}  
    \caption{Parameter study for finding REV}
\end{center}

\end{table}}%
\vspace{-0.7cm}

{\scriptsize
\begin{equation}
\label{eq:Mass_REV}
    REV  \: = \left\{\begin{array}{rcl}
         Mass\;of\; REV \;\; [kg]  & > & 1.01 \times (D_{cylinder}/D_{max}) -1.66\\
              D_{cylinder}/D_{max} & \geqslant  & 5 \\
              D_{cylinder,\; min}& \geqslant  & 5.D_{max}  
\end{array}\right.  %% <--here
\end{equation}}
\end{frame}
}


{
\usenavigationsymbolstemplate{}

\begin{frame}{REV for a widely graded soil}
\begin{figure}[H]
   \centering
\includegraphics[width=.78\linewidth]{REV.pdf} 
        \caption{Trend line of allowed REV related to maximum particle size $D_{s}$ of investigated PSD}
        \label{fig:Trendline_REV}
\end{figure}
\end{frame}
}
{
\usenavigationsymbolstemplate{}

%
% Place your slide code here
%
\begin{frame}{Packing generation based on MFBA}
Images show the insertion, the growing , the end of growing, and the relaxation phases respectively.
\vspace{0.2cm}
\begin{columns}
\begin{column}{.49\textwidth}
\centering
\hspace{0.2cm} a) \includegraphics[trim = 55mm 185mm 45mm 25mm, clip,width=.75\linewidth]{Packing_Insert_insertPack.pdf}\\
\hspace{0.2cm} b) \includegraphics[trim = 55mm 185mm 45mm 25mm, clip,width=.75\linewidth]{Packing_Insert_startGrowing.pdf}
\end{column}%
\hfill%
\begin{column}{.49\textwidth}
\centering
c) \includegraphics[trim = 55mm 185mm 45mm 25mm, clip,width=.75\linewidth]{Packing_Insert_startGrowing200.pdf}\\
d) \includegraphics[trim = 55mm 185mm 45mm 25mm, clip,width=.75\linewidth]{Packing_Insert_startGrowing300.pdf}
\end{column}%
\end{columns}

\end{frame}
}


{
\usenavigationsymbolstemplate{}

%
% Place your slide code here
%
\begin{frame}{REV based on MFBA}
\begin{figure}[H]
	\centering
	\includegraphics[width=.85\linewidth]{REV_MFBA.pdf} 
     \caption{REV size based on porosity calculations in the Packing}
        \label{REV_MFBA}
\end{figure} 

\end{frame}
}


{
\section{Quantification of homogeneity}
}
{
\begin{frame}{Quantification of Homogeneity}

	\begin{figure}[H]
\centering
\begin{tabular}{ p{1.9cm} | p{1.9cm} | p{1.9cm}}

subimage 11 \newline 
\includegraphics[trim = 22mm 15mm 20mm 10mm, clip, width=.9\linewidth]{PSD1_F13_S20_11.png}
&
subimage 12 \newline 
\includegraphics[trim = 22mm 15mm 20mm 10mm, clip, width=.9\linewidth]{PSD1_F13_S20_12.png} 
&
subimage 13 \newline 
\includegraphics[trim = 22mm 15mm 20mm 10mm, clip, width=.9\linewidth]{PSD1_F13_S20_13.png}\\

\hline 
subimage 21 \newline 
\includegraphics[trim = 22mm 15mm 20mm 10mm, clip, width=.9\linewidth]{PSD1_F13_S20_21.png} 
&
subimage 22 \newline 
\includegraphics[trim = 22mm 15mm 20mm 10mm, clip, width=.9\linewidth]{PSD1_F13_S20_22.png} 
&
subimage 23 \newline 
\includegraphics[trim = 22mm 15mm 20mm 10mm, clip, width=.9\linewidth]{PSD1_F13_S20_23.png} \\
\hline 
subimage 31 \newline 
\includegraphics[trim = 22mm 15mm 20mm 10mm, clip, width=.9\linewidth]{PSD1_F13_S20_31.png} 
&
subimage 32 \newline 
\includegraphics[trim = 22mm 15mm 20mm 10mm, clip, width=.9\linewidth]{PSD1_F13_S20_32.png} 
&
subimage 33 \newline 
\includegraphics[trim = 22mm 15mm 20mm 10mm, clip, width=.9\linewidth]{PSD1_F13_S20_33.png} \\
\hline 
\end{tabular}
	\caption{Image segmentation for calculation of $C_{H}$ (subimage number: $3 \times 3$) for PSD2 for packing with skeleton fractions $F1..3$ at $z\;=\;20\;[pixel]$}
	\label{tab:imagesegmentation}
\end{figure}

%\begin{equation}
  %\label{eq100}
   %\left\{\begin{array}{rcl}
                    %\sigma &  = & \sqrt[2]{1/N\sum_{n=1}^{N} (x_i-\mu)^2}\; = \; 0\\
                        %&    &   \\                     
                    %C_H & = & (1-\sigma^2)100\;\;[\%]\\
%\end{array}\right.  %% <--here
%\end{equation}


\end{frame}
}


\begin{frame}{Quantification of Homogeneity}
	
\begin{figure}[H]
   \centering
\includegraphics[width=.74\linewidth]{REV_CH_PSD1_F13_z.pdf} 
        \caption{Calculation of $C_{H}$ for packing with skeletal fractions $F1..3$ - The calculated COV for subimage size of $24\times24$ pixels equal to \num{8.87} indicates the excellent quality of the sample. The $C_{H}=74.31\%$ represents a stochastically homogeneous sample.}
        \label{fig:Homogeneity_calc_DEM_Section}
\end{figure}
\end{frame}

\section{Packing Model/Homogeneity/Suffusion Criterion}

\begin{frame}{Apollonian Gasket}
\begin{columns}
\begin{column}{0.49\textwidth}
\vspace{0.25cm}
\begin{figure}
  \centering
  \includegraphics[trim=9mm 53mm 7mm 50mm, clip,width=.75\linewidth]{2D_Apollonian_Gasket.pdf}
	\vspace{0.2cm}
	\caption{Densest 2D-packings for poly-disperse spheres (Apollonian Gasket)}
	\end{figure}
\end{column}%
\begin{column}{0.49\textwidth}
\begin{figure}
  \centering
\includegraphics[trim=60mm 0mm 50mm 0mm, clip, width=.75\linewidth]{3D_Apollonian_Gasket.pdf}
\caption{Densest 3D-packings for poly-disperse spheres (Apollonian Gasket)} 
\end{figure}
\end{column}
\end{columns}
        
\end{frame}


{
\usenavigationsymbolstemplate{}
\begin{frame}{New Packing Model}
	\begin{figure}[H]
\centering
\includegraphics[trim = 0mm 35mm 0mm 35mm, clip, height=7.3cm]{Packing_model_Homo_Segregated_SEMAR_FULLER_IDEAL.pdf}
\vspace{-0.4cm} 
        \caption{Schematic illustration of packing model for widely graded soils}
        \label{fig:segregated_packing_model}
\end{figure}
\end{frame}
}

{
\usenavigationsymbolstemplate{}
\begin{frame}{Packing model for any widely graded soil}
\begin{figure}
	\centering
		\includegraphics[trim=0mm 100mm 0mm 100mm, clip, width=.99\linewidth]{IHP.pdf}
	\caption{Concept of Ideal Homogeneous Packing}
	\label{fig:IHP}
\end{figure}
	
\end{frame}
}

{
\usenavigationsymbolstemplate{}
\begin{frame}{Packing model for any widely graded soil}
\begin{figure}
	\centering
		\includegraphics[trim=0mm 95mm 0mm 95mm, clip, width=.95\linewidth]{IHP_melt2.pdf}
	\caption{Concept of Ideal Homogeneous Packing}
	\label{fig:IHP}
\end{figure}
	
\end{frame}
}

%%%HERE
\begin{frame}{Suffusion criterion}

\begin{figure}[H]
   \centering
\includegraphics[trim= 0mm 50mm 0mm 50mm,clip,width=.8\linewidth]{Suffusion_Criteria_SAD.pdf} 
\vspace{-0.3cm}
        \caption{Suffusion criterion}
        \label{fig:reserved_area_PSD1}
\end{figure}
\end{frame}

\begin{frame}{Example calculation using proposed suffusion criterion (PSD1)}
Fraction F1, F6, F7 and F8 has to fill the rest of reserved areas $\Rightarrow$ they come to contact to each other

$\Rightarrow \; D_{F1}/D_{F6}=20/0.75= 26.66 > 6.46\:  \Rightarrow\:$ not filter stable 




\begin{columns}
\begin{column}{0.68\textwidth}
{\scriptsize
\centering
\begin{tabular}{|c|c|c|c|c|c|c|}
\toprule
F & $D_{F}$  & $ 3 \times D_{F}$ & RA & RA & PDF & $\Delta R_{A}$ \\\hline
[-] & [$mm$] &  [$mm$] &  [$mm^{2}$] & [\%] & [\%] & [\%]\\\hline
F1 & 20  & 60  & 6000 & 42.7 & 52.4& +9.70 \\
\textcolor{red}{F2} & 16  & 48  & 4800 & 34.2 & 21.4& \textcolor{red}{-12.8} \\
\textcolor{red}{F3} & 5   & 15  & 1500 & 10.7 & 5.10& \textcolor{red}{-5.60} \\
\textcolor{red}{F4} & 3   & 9   & 900  & 6.40 & 4.00& \textcolor{red}{-2.40} \\
\textcolor{red}{F5} & 1.3 & 3.9 & 390  & 2.80 & 2.00& \textcolor{red}{-0.80} \\
F6 & 0.75& 2.25& 225  & 1.60 & 2.00& +0.40  \\
F7 & 0.5 & 1.50& 150  & 1.10 & 7.10& +6.00  \\
F8 & 0.25& 0.75& 75   & 0.50 & 6.00& +5.50  \\\hline
   & & $\Sigma$ & 14040   & 100  & 100 & 0.0 \\
   \bottomrule
\end{tabular}
}
\end{column}
\begin{column}{0.31\textwidth}
\includegraphics[height = 5.4 cm]{Calc_PSD1_Semar2.pdf} 
\label{fig:areas_PSD1}
\end{column}
\end{columns}
\end{frame}



{
\usenavigationsymbolstemplate{}
\begin{frame}{Suffusion criterion}
\begin{figure}[H]
   \centering
\includegraphics[width=.78\linewidth]{Suff_criterion_SAD_not_log_PSD1.pdf} 
        \caption{Calculation of difference between ideal homogeneous internally stable packing with packing generated for PSD1}
        \label{fig:reserved_area_PSD1}
\end{figure}

	
\end{frame}
}
{
\usenavigationsymbolstemplate{}
\begin{frame}{Suffusion criterion}
\begin{figure}[H]
   \centering
\includegraphics[width=.78\linewidth]{Hennes_Winkler_SFT.pdf} 
\vspace{-0.2cm}
        \caption{Significance of the proposed suffusion criterion - PSD1 with different amount of fill}
        \label{fig:reserved_area_PSD1}
\end{figure}

	
\end{frame}
}

{
\usenavigationsymbolstemplate{}
\begin{frame}{Significance of Suffusion criterion}
\begin{figure}[H]
   \centering
\includegraphics[width=.8\linewidth]{KritMo_CH.png} 
\vspace{-0.3cm}
        \caption{proposed suffusion criterion - PSD1 with different amount of fill}
        \label{fig:Krit_MO}
\end{figure}

	
\end{frame}
}


\begin{frame}{Effect of homogeneity measurement in Field}
When is the measured $C_{H}$ dangerous?
\begin{figure}[H]
   \centering
\includegraphics[trim= 0mm 94mm 0mm 94mm,clip,width=.95\linewidth]{Effect_CH-Field_situation.pdf} 
\vspace{-0.3cm}
        \caption{Measured $C_{H}$ and its relationship to local and global particle transport }
        \label{fig:CH-field}
\end{figure}
\end{frame}



\begin{frame}{The main results}
\begin{enumerate}
	\item \textbf{Quantification of Homogeneity $C_{H}$} for numerical models as well as for laboratory and field situation using \textbf{image analysis} and \textbf{experimentally}
	\item \textbf{Very simple suffusion criterion} based on fraction diameters and its masses $$ f(D_{i}/\sum(D_{i}),M)$$
	\begin{itemize}
		\item which considers Homogeneity
		\item which identifies the dominant matrix
	\end{itemize} 
	\item \textbf{packing model based on loosest state} of the soil for reproducibility of experimental procedure
\end{enumerate}
\end{frame}







{\Large
\plain{Questions?}
}
\begin{frame}[allowframebreaks]{References}

  \bibliography{Phd}
  %\bibliographystyle{abbrv}
  \bibliographystyle{agsm}

\end{frame}

\begin{frame}{equation of motion}

For each sphere, the equation of motion is written, and the resulting two equations can be combined to obtain the governing differential equation for the penetration x, 
\begin{align}
\ddot{x}+(D/m_{r})\dot{x}+(K/m_{r})x=0
\label{equ:motion}
\end{align}

where 
$
m_{r}=\dfrac{m_{1}m_{2}}{m_{1}+m_{2}}
$
is the reduced mass. Equation of motion is solved for the penetration $x$ in the usual fashion to yield,

\begin{align}
x(t)   \: = \: C_{1}e^{-\mu t}\cos(\Omega t) + C_{2}e^{-\mu t}\sin(\Omega t)\\
\Omega \: = \: \sqrt{\omega_{0}^{2} - \mu^{2}}\\
\omega_{0}^{2} \: = \: K/m_{r} , \mu\: = \:D/2m_{r}
\end{align}

the natural frequency is $f\: = \:\Omega / 2\pi$ Thus, the collision time tc can be approximated as $t_{c} \: = \: 1/2f \: = \:\pi / \sqrt{\omega_{0}^{2} - \mu^{2}}$

\end{frame}
\begin{frame}{Proof for suffusion tests }

%Proof for Prof. Witt for different effect of homogeneity on particle transport.

\begin{table}[b]

{
\scriptsize
\begin{center}
\begin{tabular}{lcccccc}
Hydraulic     & SRT $q*10^{-3}$ & ALS $q*10^{-3}$  & DLS $q*10^{-3}$ \\
Gradient (i)  & [$m.s^{-1}$]    & [$m.s^{-1}$]    & [$m.s^{-1}$] \\\hline
0.03          & $5.5\pm 0.02 $  & $5.18\pm 0.01$  & $4.6\pm 0.02$ \\
0.05          & $18\pm 0.02 $  & $17.3\pm 0.01$   & $15.12\pm 0.02$ \\
0.08          & $22\pm 0.02 $  & $21.43\pm 0.02$  & $19.66\pm 0.04$ \\
0.1          &  $26\pm 0.06 $  & $25.55\pm 0.04$  & $24.32\pm 0.06$ \\\hline
\end{tabular}
    \label{tab:Values of Volume Flux}  
    \caption{Values of volume Flux for different samples at different hydraulic gradient - }
\end{center}
}%
\end{table}

Among the three samples, the mixed SRT
has the highest permeability. This indicates that fluid flow faster in mixed sample
than layered segregated media.
\end{frame}


{
\usenavigationsymbolstemplate{}

%
% Place your slide code here
%
\begin{frame}[fragile]{Relevance of the study} 

In-situ particle distributions are highly dependent on placement parameters during construction 
\begin{itemize}
\item moisture content of the soil
\item drop height\item geometry
\item particle coarseness
\item relation between particle diameters
\item dynamical vibrations
\end{itemize}
After \cite{KenneyWestland:1993} the poorly graded soils are also very vulnerable to segregation\\

\alert{\textbf{Is the assumption of soil homogeneity a suitable assumption????????} }
\end{frame}
}

\begin{frame}[fragile]{Relevance of the study}
\begin{figure}[h]
	\centering
		\includegraphics[trim = 1mm 13mm 4mm 6mm, clip, width=.80\linewidth]{Kenney_westland_1993.png}
		       \caption{Initial and segregated PSDs;  
        curve 1 initial PSD;
        curve 2 and 2' relatively segregated;
        curve 3 and 3' perfect segregated after \citep{KenneyWestland:1993}}
        \label{kenwestseg93}
\end{figure}
\end{frame}


\begin{frame}%{Energy dissipation in the sample}
\begin{figure}[H] 
   \centering
\includegraphics[trim = 10mm 6mm 10mm 6mm, clip,width=.96\linewidth]{Schematic_suffusion_test_dwg-Model.pdf} 
        \caption{Schematic illustration of total head surface for suffusion tests}
        \label{fig:suffusion_test_total_head_loss}
\end{figure}

\end{frame}


{
\usenavigationsymbolstemplate{}

%
% Place your slide code here
%
\begin{frame}[fragile]{Particle arrangement $\;\longrightarrow\;$  behavior of all materials}
\begin{figure}[H]

   \centering
     \includegraphics[trim = 0mm 3mm 0mm 0mm, clip,width=.91\linewidth]{DCM_TM.png}
        \caption{Sequential Fill Test (SFT) - soil matrix DCM-TM}
        \label{fig:EXP-SFT2}
\end{figure}

\end{frame}
}
%%%%%%%%%%%%%%%%%%%%%%%%%%%%%%%%%%%%%%%%
{
\usenavigationsymbolstemplate{}

%
% Place your slide code here
%
\begin{frame}[fragile]{Particle arrangement $\;\longrightarrow\;$  behavior of all materials}
\begin{figure}[H]

   \centering
     \includegraphics[trim = 0mm 3mm 0mm 0mm, clip, width=.91\linewidth]{TM_DFM.png}
        \caption{Sequential Fill Test (SFT) soil matrix TM-DFM}
        \label{fig:EXP-SFT3}
\end{figure}

\end{frame}

\begin{frame} {Linear Spring-Dashpot Model for Two Interacting Spheres}
\begin{columns}[T,onlytextwidth]
\column{0.62\textwidth} 
\vspace{0.32cm}
Consider a sphere of mass $m_{1}$ and diameter $d_{1}$ whose collisional interaction with another sphere (mass $m_{2}$, diameter $d_{2}$) is idealized by a dashpot and spring as depicted in Fig. \ref{fig:dp} Define the penetration distance $x$ as:
\begin{align}
x\:=(x_{1}-d_{1}/2)-(x_{2}-d_{2}/2)=\\
x\:=x_{1}-x_{2}+1/2(d_{1}+d_{2})
\end{align}
Hence the interpenetration velocity and acceleration is given by
\begin{align}
\dot{x}\:=\dot{x_{1}}-\dot{x_{2}}\\
\ddot{x}\:=\ddot{x_{1}}-\ddot{x_{2}}
\end{align}


\column{0.32\textwidth}
\begin{figure}[t]
\centering
\includegraphics[width=7cm, height=4.8cm, keepaspectratio]{Dashpot-spring-model_normal-forces.png}
\caption{Sphere interaction modeled as a spring and dashpot \cite{goniva2010open}}
\label{fig:dp}
\end{figure}

\end{columns}
\end{frame}
}


{
\usenavigationsymbolstemplate{}

\begin{frame}{Quantification of Homogeneity}
\begin{figure}[H]
\centering
\begin{tabular}{ p{2.3cm} | p{2.3cm} | p{2.3cm}}
subimage 11 \newline 
\includegraphics[trim = 70mm 120mm 70mm 105mm, clip, width=0.70\linewidth]{Slice_Mid_3x3/11.pdf}
&
subimage 12 \newline 
\includegraphics[trim = 70mm 120mm 70mm 105mm, clip, width=0.70\linewidth]{Slice_Mid_3x3/21.pdf} 
&
subimage 13 \newline 
\includegraphics[trim = 70mm 120mm 70mm 105mm, clip, width=0.70\linewidth]{Slice_Mid_3x3/31.pdf} \\\hline 
subimage 21 \newline 
\includegraphics[trim = 70mm 120mm 70mm 105mm, clip, width=0.70\linewidth]{Slice_Mid_3x3/12.pdf}
&
subimage 22 \newline 
\includegraphics[trim = 70mm 120mm 70mm 105mm, clip, width=0.70\linewidth]{Slice_Mid_3x3/22.pdf} 
&
subimage 23 \newline 
\includegraphics[trim = 70mm 120mm 70mm 105mm, clip, width=0.70\linewidth]{Slice_Mid_3x3/32.pdf}  \\\hline 
subimage 31 \newline 
\includegraphics[trim = 70mm 120mm 70mm 105mm, clip, width=0.70\linewidth]{Slice_Mid_3x3/13.pdf} 
&
subimage 32 \newline 
\includegraphics[trim = 70mm 120mm 70mm 105mm, clip, width=0.70\linewidth]{Slice_Mid_3x3/23.pdf} 
&
subimage 33 \newline 
\includegraphics[trim = 70mm 120mm 70mm 105mm, clip, width=0.70\linewidth]{Slice_Mid_3x3/33.pdf} \\\hline 
\end{tabular}
	\caption{Image segmentation for calculation of $C_{H}$ of pores (sub-image number: $3 \times 3$) for lattice packing at $z\;=\;10\;pixel$ }
	\label{tab:imagesegmentationLattice}
\end{figure}
\end{frame}
}


\begin{frame}
%%%%%%%%%%%%%%%%%%%%%%%%%%%%%%%%%%%%%%%%%%%%%%%%%%%%%%%%%%%%%%%%%%%%%%%%%%%%%%%%%%%%%%%
\begin{table}[H]
{
\tiny
\begin{tabular}{L{3.1cm}  L{2.7cm}  L{4.5cm}}
\textbf{ Parameter matrix} & \textbf{Normalized matrix }& \textbf{Vector of parameter and its mean value}\\
 \hline \hline
\[PND=
\begin{vmatrix}
4.00 & 3.00 & 3.00 \\
3.00 & 5.00 & 4.00 \\
4.00 & 2.00 & 2.00 \\
\end{vmatrix}\Rightarrow
\]
& 
\[\hat{PND} =
\begin{vmatrix}
0.80 & 0.60 & 0.60 \\
0.60 & 1.00 & 1.00 \\
0.80 & 0.40 & 0.40\\
\end{vmatrix}\Rightarrow
\]
&
% 0.1 & 0.2 & 0.31 & 0.4&0.5& 0.10& 0.12 & 0.32 & 0.2 \\

\[ X_{1}=
\begin{vmatrix}
0.80 & 0.60 & 0.80 & 0.60 & 1.00 \\& 0.40 & 0.60 & 1.00 & 0.40
\end{vmatrix}
\]
\[\Rightarrow \mu_{X_{1}}\;=\; 0.69\]\\
% 2nd Row
\[n=
\begin{vmatrix}
0.45 & 0.43 & 0.62 \\
0.31 & 0.42 & 0.26 \\
0.40 & 0.39 & 0.28 \\
\end{vmatrix}\Rightarrow
\]
& 
\[\hat{n} =
\begin{vmatrix}
0.73 & 0.69 & 1.00 \\
0.51 & 0.68 & 0.42 \\
0.65 & 0.63 & 0.46 \\
\end{vmatrix}\Rightarrow
\]
& 
% 0.1 & 0.2 & 0.31 & 0.4&0.5& 0.10& 0.12 & 0.32 & 0.2 \\

\[ X_{2}=
\begin{vmatrix}
0.73 & 0.69 & 1.00 & 0.51 & 0.68 \\& 0.42 & 0.65 & 0.63 & 0.46 
\end{vmatrix}
\]\[\Rightarrow \mu_{X_{2}}\;=\; 0.64\]\\
%third row
\[D_{e}=
\begin{vmatrix}
10.29 & 14.92 & 15.89 \\
14.68 & 9.21  & 11.37 \\
9.24  & 19.65 & 18.96 \\
\end{vmatrix}\Rightarrow
\]
& 
\[\hat{D_{e}} =
\begin{vmatrix}
0.52 & 0.76 & 0.80 \\
0.75 & 0.47 & 0.58 \\
0.47 & 1.00 & 0.97 \\
\end{vmatrix}\Rightarrow
\]
&
\[ X_{3}=
\begin{vmatrix}
0.52 & 0.76 & 0.80 & 0.75 & 0.47 \\& 0.58 & 0.47 & 1.00 & 0.97 
\end{vmatrix}
\]\[\Rightarrow \mu_{X_{3}}\;=\; 0.70\]\\ \hline 
% 4th row
\[
COV(X_{1},X_{3})\:=\: \sum_{ i=1}^{N=3}(x_{i}-\mu_{X_{i}})(x_{i+1}-\mu_{X_{i+1}})/N \; \Rightarrow
\]
\[
\sum(COV(X_{1},X_{3})= 0.309 \; \Longrightarrow  C_{H} = (1-sum(COV(X_{1},X_{3})) \times 100\;  \Longrightarrow \; C_{H} \:=\: 74.31 \% 
\]
&

&
\[COV(X_{1},X_{3})\:=\:
\begin{vmatrix}
0.051 &	0.022	& 0.042\\
0.022 &	0.030	& 0.028\\
0.042 &	0.028	& 0.041\\
\end{vmatrix}\Downarrow
\]\\
\\ \hline

\end{tabular}
	\caption{Calculation method of C_{H}}
	\label{ch:CH_calc_diag}
	}
\end{table}
%%%%%%%%%%%%%%%
\end{frame}



\begin{frame}

\begin{figure}[H]
   \centering
\includegraphics[width=.92\linewidth]{REV_CH_lattice.pdf} 
        \caption{$C_{H}$ calculations for almost perfect homogeneous sample (see Figure \ref{tab:imagesegmentationLattice}}
        \label{fig:Homogeneity_calc_lattice_Section}
\end{figure}

\end{frame}

\section*{Suffusion tests}


{
\usenavigationsymbolstemplate{}
\begin{frame}{Suffusion tests (ST)}
\begin{figure}[H]

   \centering
\includegraphics[trim = 2mm 85mm 2mm 85mm, clip, width=.95\linewidth]{suffusion_test_apparatus.pdf} 
        \caption{Principle sketch of the suffusion test procedure }
        \label{fig:suff_test_device}
\end{figure}
\end{frame}
}


%\metroset{titleformat frame=smallcaps}

\begin{frame}{Result of Suffusion Tests (ST)}


\begin{figure}
	\centering
		\includegraphics[trim= 0mm 5mm 0mm 15mm, clip, width=0.92\textwidth]{Scatterplot_ST_EM2.pdf}
		\vspace{-0.1cm}
	\caption{High variation of the eroded mass in different suffusion tests}
	\label{fig:scatterplot_ST_EM}
\end{figure}


\end{frame}


\begin{frame}
\begin{figure} [H]
   \centering
\includegraphics[trim = 4mm 5mm 4mm 5mm, clip,width=.98\linewidth]{Non-homogeneous_time-differ_i.pdf} \caption{Contours of total head for specimen A - non-homogeneous}       
\end{figure}
\end{frame}

{
\usenavigationsymbolstemplate{}

%
% Place your slide code here
%
\begin{frame}
\begin{figure}
   \centering
\includegraphics[trim = 4mm 5mm 5mm 4mm, clip,width=.98
\linewidth]{homogeneous_time-differ_i.pdf} 
\caption{Contours of total head for specimen B - stochastically homogeneous}
\end{figure}
\end{frame}
}

\begin{frame}
		  \begin{center}\begin{chronology}[2]{2008}{2015}{0.85\textwidth}
\event[\decimaldate{1}{2}{2008}]{\decimaldate{22}{1}{2014}}{132 different tests }
\event[\decimaldate{1}{2}{2008}]{\decimaldate{22}{1}{2011}}{52 Suffusion tests}
\event[\decimaldate{27}{4}{2014}]{\decimaldate{30}{8}{2014}}{\alert{Suffusion Criterion}}
\end{chronology}
\end{center}
\end{frame}


\begin{frame}
\label{{img2_Scaterplot}}
\begin{figure}
	\centering
		\hyperlink{suffusion}{  \includegraphics[trim = 0mm 3mm 0mm 0mm, clip, width=.98\linewidth]{Scatterplot_ST_EM2.pdf}}
	\caption{Scatter plot eroded masses }
	\label{fig:Fig4_ST_4L}
\end{figure}

\end{frame}

\begin{frame}\label{img1_ST_setup}
	\begin{figure}
	\centering
		\hyperlink{suffusion}{   \includegraphics[trim = 0mm 3mm 0mm 0mm, clip, width=.98\linewidth]{Fig4_ST_4L.pdf}   }
\end{figure}
\end{frame}

\begin{frame}{Input parameter for DEM modeling}
\begin{table}[H]
{\scriptsize
\begin{center}
\begin{tabular}{lccc}
Parameter & Symbol & Units & Value\\\hline
Particle density & $\rho$ & $kg/m^{3}$ & 2500 \& 2650\\
Young's Modulus & $E$ & $GPa$ & 50\\
Coefficient of Restitution & $\varsigma$ & - & 0.3\\
Coefficient of inter-particle friction & $\mu$ & - & 0.1\\
Poison ratio & $\nu$ & - & 0.3\\
Length of timestep & $\Delta\:t$ & sec. & $10^{-7}$\\\hline
\end{tabular}
     \caption{DEM model parameters for numerical simulations.}
    \label{tab:DEM_model_parameter}  
\end{center}
}%
\end{table} 

{\scriptsize
\begin{table}[H] 
\centering
\begin{tabular}{lcccccc}
Fraction & F1 & F2 & F3 & F4 & F5 & F6  \\\hline
$D/Dmax$ PSD2 [-] & $1$ & $0.8$ & $0.315$ & $0.2$ & $0.1$ & $0.05$ \\
$D/Dmax$ PSD3 [-] & $1$ & $0.8$ & $0.25$ & $0.15$ & $0.065$ & $0.0375$ \\\hline 
\end{tabular}
\caption{Different fractions of the normalized PSDs for numerical simulations } 
\label{Tab_diff_fraction_norm_PSD}
\end{table}
}

\end{frame}




\begin{frame}{investigated PSDs for numerical modeling}
\begin{figure}[H]
   \centering
\includegraphics[trim = 0mm 3mm 0mm 17mm, clip, width=.9\linewidth]{PSD2_PSD3_Semar.pdf} 
        \caption{Normalized PSD used in DEM simulations}.
        \label{fig:PSD1PSD2_norm_Simu}
\end{figure}
\end{frame}


{
\usenavigationsymbolstemplate{}
%
% Place your slide code here
%




\begin{frame}{Experimental results for SFT}
\begin{figure}[H]
   \centering
\includegraphics[trim = 0mm 4mm 0mm 5mm, clip, width=.8\linewidth]{SFT_PSD2_PSD3.pdf} 
        \caption{SFT result for PSD2 and PSD3, $ h_{r}: $ relative height $n:$ porosity}
        \label{fig:SFT_result_PSD2_PSD3}
\end{figure}

\end{frame}

\begin{frame}{Experimental quantification of Homogeneity}

\begin{figure}[H]
   \centering
\includegraphics[width=.8\linewidth]{CH_EXPerimental.png} 
\vspace{-0.3cm}
        \caption{Experimental coefficient of homoegeneity $C_{H}$}
        \label{fig:reserved_area_PSD1}
\end{figure}

\begin{figure}[H]
   \centering
\includegraphics[width=.45\linewidth]{H_Field.png} 
\vspace{-0.3cm}
        \caption{Equivalent height of the sample taken from field}
        \label{fig:reserved_area_PSD1}
\end{figure}
\end{frame}
{
\usenavigationsymbolstemplate{}
\begin{frame}{Summary}
\begin{figure}[H]
   \centering
\includegraphics[ width=\textwidth]{findings1.png} 

\end{figure}
\break

	
\end{frame}
}

{
\usenavigationsymbolstemplate{}
\begin{frame}{Summary}
\begin{figure}[H]
   \centering
\includegraphics[ width=\textwidth]{findings2.png} 

\end{figure}
\break

	
\end{frame}
}

\end{document}
